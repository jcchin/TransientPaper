\documentclass[heading.tex]{subfiles} 
\begin{document}

% \tableofcontents

\section{Background and Motivation}
Historically, there have been completely separate code-bases for design space exploration and transient engine analysis. The capability to create flexible "low fidelity" models is ideal in the earliest  conceptual design stages, where large design spaces can be explored relatively quickly. As the engine design starts to become more concrete, higher fidelity models are developed. These higher detail models are intrinsically more sensitive to design tweaks, and in general take longer to setup. They inherently require more stringently defined sets of boundary conditions and a specific configuration. A large variation in the baseline design could potentially render entire higher fidelity analyses obsolete. Due to this weakness in adaptability, higher fidelity models are generally not created until the low-fidelity design has fully matured. This friction can partly be attributed to the differences in toolstes when transitioning to higher fidelity models. 

\section{Introduction}
Here goes my story.This is the introduction.
Yahee.

\section{Rest of the Paper}
Write the rest of your paper using the normal \LaTeX\ stuff you like to
use.

\section{Conclusions}
This is conclusions section it's quite handy.


\end{document}

%%% Local Variables: 
%%% mode: latex
%%% TeX-master: t
%%% End: 
